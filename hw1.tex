\documentclass[14pt, letterpaper, oneside]{extarticle}
\usepackage{graphicx}
\usepackage{float}
\usepackage[margin=1in]{geometry}
\usepackage{parskip}
\usepackage{listings}
\usepackage{hyperref} 
\usepackage{array}
\usepackage{tabularx}
\usepackage{amsmath, amssymb}
\newcolumntype{Y}{>{\centering\arraybackslash}X}

\begin{document}
% chktex-file 44

\begin{titlepage}
    \vspace*{\fill} %
    \centering
    {\Huge \textbf{HW 1} \par}
    {\vspace{2cm}}
    {\LARGE By Tony Song (ls2272) \par}
    {\vspace{2cm}}
    {\LARGE September 5th, 2025 \par}
    \vspace*{\fill}
\end{titlepage}

\newpage
\textbf{(1)}

(a) The maximum number of links between four hosts is 6 ($1 + 2 + 3 = 6$).
Thus, such a network is not possible.

(b) Total degrees $= 2 + 1 + 1 + 1 = 5$

Edges $= \frac{\text{Total degrees}}{2} = \frac{5}{2} = 2.5$

The number of edges must be a whole number. Thus, such a network is not possible.

\textbf{(2)}

(a) Node D has degree 4. Node J has degree 6.
Every other node (A, B, C, E, F, G, H, I) has degree 1.

(b) The graph would have 4 connected components if node D was removed.
The graph would have 6 connected components if node J was removed.
The graph would still have 1 connected component if any other node (A, B, C, E, F, G, H, I) was removed.

(c) Nodes D and J have the property that the maximum distance from this bank to any other bank is at most 2.
They have this property because they are centrally located in this graph.

(d) No, this is not always true. A centrally located node is one that has the shortest total distance from other nodes.
Having the highest degree is having the most immediate neighbours. A node with the highest degree can be at the end of the network,
where it would have longer distances with other nodes, making it not centrally located.

% insert drawn graph

\textbf{(3)}

(a) % insert drawn graph

(b) Yes % insert drawn graph

(c) Yes % insert drawn graph

\textbf{(4)}

\end{document}